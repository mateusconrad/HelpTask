\chapter{Projeto de Pesquisa} \label{chap:ResearchPlan}

\section{Tema} \label{sec::Theme}
Análise e desenvolvimento de um aplicativo mobile para realizar e atender chamados de suporte interno.

\subsection{Delimitação do Tema} \label{subsec::ThemeDelimitation}
A delimitação do projeto se dará a análise e desenvolvimento de uma aplicação mobile de chamados de suporte interno - help desk -  em uma empresa de software aplicando o guia de boas práticas da ITIL. A pesquisa será desenvolvida pelos acadêmicos Bruno Leonardo Giehl e Mateus Augusto Conrad no período compreendido entre Agosto e Dezembro de 2019, em parceria com a empresa Abase Sistemas e Soluções Ltda, na qual a pesquisa será desenvolvida.

%objetivos
\section{Objetivos} \label{sec:objective}
O objetivo visa identificar, de forma abrangente, a proposta para solução do problema proposto e o mesmo deve ser descrito com verbos no infinitivo \citep{lovato_2007}.

\subsection{Objetivo Geral}
O objetivo geral do trabalho é desenvolver um aplicativo para padronizar o processo de abertura de chamados de suporte internos de uma empresa de TI.
%objetivos específicos
\subsection{Objetivos Específicos}
\begin{enumerate}
    \item Fazer a análise e desenvolvimento de um app para chamados de suporte.
    \item Usar um banco de dados em tempo real para gerenciamento dos dados
    \item Fazer os testes de Software no aplicativo 
    \item Disponibilizar o arquivo para instalação do app.
\end{enumerate}

\section{Justificativa}\label{sec:justification}
De acordo com \citet{marconi_lakatos_2010}, a justificativa possui o objetivo de apresentar o porquê de algo estar sendo realizado. A mesma faz uso de recursos teóricos e práticos para assim conseguir alcançar seu objetivo. A Justificativa serve, principalmente para justificar a pesquisa e convencer o leitor ou orientador da mesma.

O trabalho pode ser justificado a partir de dois principais fatores: o porquê desenvolver mobile, o porquê usar de conceitos de gerenciamento de TI e também o motivo de unir esses dois fatores. 

O avanço da tecnologia aplicada em smartphones e a maneira de como as empresas modernas estão se comportando está mudando constantemente. A partir do momento que se há um negócio em desenvolvimento ou se está começando a desenvolver algo, ter acesso o acesso do mesmo por um aplicativo para smartphone é fundamentalmente considerável, isso devido a quantidade de usuários nessa plataforma. De acordo com uma pesquisa realizada pelo IBGE, o acesso a internet já era 69\% proveniente de smartphones em 2016, ou seja, além do crescente número de acessos por smartphones, diminui o número de acessos por computadores.

No quesito de desenvolvimento mobile, há discussões envolvendo o uso de desenvolvimento de páginas web responsivas ao invés de aplicativos instalados no dispositivo do usuário. Apesar de muitas vezes aplicativos sites responsivos oferecerem a mesma funcionalidade, ter um aplicativo pode diminuir o tempo de espera de carregamento de páginas, justamente pelo fato de não necessitar de         tantas requisições para o servidor da aplicação. Outro fator contra o uso de páginas web é o uso de uma conexão de qualidade ruim, podendo prejudicar a experiência do usuário com o serviço, algo que pode ser desviado com o uso de um aplicativo.
\\

No que diz respeito a gerenciamento de TI, observa-se muitas vezes problemas relacionados com a falta de normalização e padronização de atividades, podendo essas atividades estarem ligadas ao desenvolvimento, suporte, entre outros. Para fazer uso de conceitos de gerenciamento de TI, podem ser achados diversos guias como ITIL, COBIT e COSO, no entanto não é possível afirmar qual é o melhor framework para aplicação, isso corresponde a afinidade do gestor com cada ferramenta ou da situação atual do negócio. Também é possível procurar uma noção geral de cada guia e definir práticas próprias para gerenciamento.

No contexto do tema do trabalho, procura-se usar do guia ITIL para criar uma solução de gerenciamento de chamados de suporte interno para uma empresa de tecnologia da informação. Então, apresentando os motivos principais da aplicação de cada objeto incluído neste trabalho, pode-se unir o gerenciamento da tecnologia da informação com o desenvolvimento mobile, para assim criar um aplicativo de gerenciamento de chamados, aplicando o mesmo em uma empresa para testar o software.

\section{Problema} \label{sec::Problem}
Como um aplicativo pode padronizar e agilizar o processo de abertura de chamados de suporte internos de uma empresa?

\section{Hipóteses} \label{sec::Hypothesis}
\begin{enumerate}
    \item O aplicativo desenvolvido pode definir uma padronização para os chamados de suporte interno na empresa.
    \item O aplicativo desenvolvido permite visualizar estatísticas de chamados realizados.
\end{enumerate}

\section{Metodologia} \label{sec:Methodology}
De acordo com \citet{lovato_2007}, a metodologia da pesquisa pode ser vista como uma ciência que, ao ser utilizada por pesquisadores com o objetivo de alcançar suas teorias, ou seja utiliza forma para alcançar teorias, adicionam conhecimento em um objetivo.

No entanto, para \citet{marconi_lakatos_2010}, a metodologia científica se especifica em compreender a maior quantidade de itens possível, pois somente responde a um  tempo às questões: como, com o que, onde e quando.

\subsection{Métodos de Abordagem}
Os métodos são responsáveis pelo raciocínio utilizado no desenvolvimento da pesquisa, são procedimentos gerais que guiam o desenvolvimento de uma pesquisa científica. São métodos de abordagem o método indutivo, dedutivo, hipotético-dedutivo e dialético \citep{metodologia_maria_andrade}.

Nota-se que, se a premissa maior for considerada falsa, a conclusão não terá validade a questão fundamental da dedução está na relação lógica que deve ser estabelecidas entre as proposições apresentadas, a fim de não comprometer a validade da conclusão \citep{mezzaroba_monteiro_2014}.

Como métodos de abordagem, foram identificados como mais adequados os métodos de abordagem dedutiva e qualitativa. Na abordagem dedutiva, se parte de leis e princípios para ter a capacidade de predizer a ocorrência de fenômenos particulares. Já na abordagem qualitativa, consiste em partir do princípio de estudar  um evento isolado, tendo como principal característica um caráter exploratório \citep{lovato_2007}.

A pesquisa bibliográfica é o meio utilizado para fundamentar o trabalho através de referências manuais e informatizadas no que diz respeito ao embasamento teórico, atribuindo, assim, conceitos adquiridos através de trabalhos, obras e publicações inerentes ao contexto e que colaboram com a proposta da pesquisa \cite{lovato_2007}.

Foi possível fazer uso da abordagem qualitativa partir do momento em que aplicou-se algo que já existe em literatura científica, ou seja, usou-se do conhecimento que já existe para desenvolver o trabalho.

\subsection{Procedimentos}
Os métodos de procedimento tem como papel explicar os objetos menos aplicados da pesquisa e se relacionam especificamente com a constituição das etapas da pesquisa. Os métodos de procedimento são fundamentados por procedimentos de ação, como a coleta de dados, a análise de dados e a interpretação dos resultados obtidos \citep{lovato_2007}.

Como métodos de procedimento, identificou-se a necessidade do uso da pesquisa bibliográfica para obter conhecimentos e referenciais sobre a área de negócio aplicada, gerenciamento da tecnologia da informação com alinhamento e estratégias de sistemas, desenvolvimento de aplicações mobile, bancos de dados não relacionais em tempo real.

\subsection{Técnicas}
De acordo com \cite{marconi_lakatos_2010}, as técnicas consistem num composto de métodos a serem utilizados para a elaboração da pesquisa a ser realizada ou como um modo apropriado de se investigar sistemática. Nesse caso, as técnicas se referem principalmente às etapas de produção do aplicativo, de como será planejado e desenvolvido e quais tecnologias serão usadas.

Na etapa de análise do software será utilizada a ferramenta “draw.io” para ilustrar os diagramas de processo e caso de uso. Como a base de dados será a partir de um banco não relacional, não se faz necessário um modelo normalizado para ilustrar as entidades e relacionamentos do banco, no entanto será ilustrado um modelo baseado no padrão proposto por Peter Chen, no qual se exibe as entidades e as classes do projeto. Além disso, será elaborado um backlog do produto para identificar o escopo de o que deve ser desenvolvido.  

Como técnica de codificação da aplicação, será usada a IDE Android Studio, uma ferramenta criada e mantida pela Google para desenvolvedores android. Como Linguagem de programação, será usada a linguagem Dart juntamente com o framework Flutter. Já como banco de dados será usado o Firebase, o qual tem suporte a integração com Flutter e Android Studio, e além de banco de dados pode ser usado como BaaS. Como plataforma de versionamento de código será usado o Github, onde é possível registrar o histórico de edições do código feito e editado por cada membro do projeto, registrar problemas e disponibilizar o repositório de forma open source.

\subsection{Validação das Hipóteses}
A validação das hipóteses trata de como os envolvidos no trabalho farão para corroborar as hipóteses propostas. Neste  trabalho, as hipóteses tratam de definir uma padronização dos chamados de suporte e também a visualização de estatísticas sobre os chamados realizados.

Para comprovar a hipótese que trata de definir uma padronização dos chamados de suporte é necessário contextualizar a situação atual de como os chamados de suporte funcionam. Atualmente, os chamados de suporte são realizados a partir de mensagens diretas com o responsável pela infraestrutura da empresa ou por e-mail. Então o uso do aplicativo desenvolvido poderia padronizar a forma que os chamados são realizados.

No que diz respeito para comprovar a hipótese que trata de visualizar estatísticas dos chamados realizados, o que se propõe é a especificação dos tipos de chamados quando um é realizado, tempo médio para fechamento dos chamados, então, a partir dessas informações gerar gráficos de estatísticas.
  
 \subsection{Orçamento e Cronograma}  
Na etapa inicial de projeto, foi elaborado um cronograma e também um orçamento para o trabalho em questão, disponíveis respectivamente como apêndices I e II deste documento. 
%table example
