\chapter*{Resumo} \label{chap:intro}
Nos dias atuais, empresas de todos setores do mercado se utilizam de alguma forma de infraestruturas e/ou serviços de TI para conduzir seu negócio. Isso se deve principalmente pela capacidade de aprimoramento do negócio através da implementação de ERP’S e demais ferramentas, capazes de controlar o fluxo de negócio, além de apresentar dados concretos e em tempo real acerca do negócio. A infraestrutura de TI necessita de constante suporte para que o downtime em casos de falhas/erros nos equipamentos necessários para o pleno funcionamento da operação cause o mínimo impacto possível, e, por esta razão, empresas de médio a grande porte empregam sua própria equipe de suporte, costumeiramente operando como uma central de serviços. Sendo assim, o presente trabalho propõe o desenvolvimento de uma aplicação mobile para sistemas operacionais Android, para realizar abertura e atendimento de chamados de suporte de TI, visando facilitar e padronizar o processo de atendimento utilizando-se ainda do guia de ITIL para boas prática em TI, bem como a geração de gráficos para facilitar a visualização de informações recorrentes. A presente pesquisa e desenvolvimento ocorreram tendo como objeto de pesquisa  o ambiente do setor de suporte de TI da empresa Abase Sistemas Ltda. O desenvolvimento da aplicação é tratada como a principal problemática da presente pesquisa, na qual foram analisadas as metodologias já existentes na empresa, para que assim se avaliasse as reais necessidades e funcionalidades necessárias para propor uma solução capaz de padronizar o suporte. A aplicação apresenta níveis de acesso para atendentes e usuários, permitindo criar chamados, e acompanhar o status do chamado, tendo seu desenvolvimento baseado na linguagem Dart acrescida do framework Flutter, bem como utilização de banco de dados não-relacional Firebase. Através da presente pesquisa, foi possível desenvolver uma aplicação seguindo os requisitos levantados através da avaliação do setor de TI da empresa, bem como apresentar informações úteis aos usuários através da geração de gráficos.